%%%%%%%%%%%%%%%%%%%%%%%%%%%%%%%%%%%%%%%%%%%%%%%%%%%%%%%%%%%%%%%%%%%%%%%%
% Preamble
%%%%%%%%%%%%%%%%%%%%%%%%%%%%%%%%%%%%%%%%%%%%%%%%%%%%%%%%%%%%%%%%%%%%%%%%
\documentclass[12pt]{article}
%
% Packages and other includes
% Pagination
\usepackage[letterpaper, margin=1in]{geometry}
%
% Graphics, floats, tables
\usepackage{graphicx, color, float, array}
\graphicspath{{image/}}
%
% Fonts
\usepackage[T1]{fontenc} % best for Western European languages
\usepackage{lmodern} % Latin Modern instead of CM
\usepackage{textcomp} % required to get special symbols
%
% Math
\usepackage{amsmath, amssymb}
\usepackage{enumerate}
\usepackage{braket}
% 
% Hyperlinks
\usepackage[colorlinks,linkcolor={red},citecolor={blue},
urlcolor={blue}]{hyperref} 
%
% Definitions and settings
% Paragraph indent and spacing
\setlength{\parskip}{0.4\baselineskip}
\setlength{\parindent}{0in}
%
% Math mode version of "r" column type (requires array package)
\newcolumntype{R}{>{$}r<{$}}
% Title, authors, date
\title{\textbf{Exam 3 Study Guide}}
\date{\today}

\begin{document}

\maketitle 

This is a checklist based on the lecture and textbook materials. It is not
expected to be an all encompassing study guide and provides a guideline for
your studies.

\begin{Large}
  \textbf{Newer Material}
\end{Large}

\textbf{Chapter 11: Bonding Theories}

\begin{itemize}
\item Valence Bond Theory - hybridized orbitals corresponding to the
  electronic arrangement
\item Molecular Orbital Theory - advantages and disadvantages
\item Nuances between Valence Bond Theory and Molecular Orbital Theory
\item Definition of a bond in terms of atomic orbitals
\item Sigma ($\sigma$) and Pi ($\pi$) bonds - how they are formed
\item Bonding and antibonding orbitals
\item Filling in the electrons for the molecular orbitals of diatomic atoms
\end{itemize}

\textbf{Chapter 12: Liquids and Solids}

\begin{itemize}
\item Intramolecular and intermolecular forces - what are the difference?
\item Types of intermolecular forces
\item Textbook definition of dispersion vs Prof. Nguyen's manuscript to
  describe dispersion
\item Ranking boiling point, melting point, vapor pressure, and viscosity
\item Phase changes and heating curve - calculating amount heat
\item Liquids boil at what vapor pressure assuming that the liquid is at
  ground level
\item Phase diagrams - triple point, critical point, melting, sublimation,
  and vaporization
\item Classification of solids
\item Types of unit cells - simple cubic, body-centered cubic, and face-centered
  cubic
\end{itemize}

\textbf{Chapter 13: Solutions}

\begin{itemize}
\item Different concentration units - molality, percent by mass, mole fraction,
  and molarity
\item Understand how to convert from one concentration unit to another
\item Colligative properties (nonelectrolyte and electrolyte) - boiling point elevation,
  freezing point depression, and osmotic pressure
\item Concept of vapor pressure lowering due to presence of solute
\end{itemize}

\begin{Large}
  \textbf{Material from Exams 1, 2, and 3}
\end{Large}

\textbf{Chapter 1: Matter and Energy}

\begin{itemize}
  \setlength\itemsep{0em}
\item Classification - pure substance and mixture
\item Different states of matter and its properties - solid, liquid, and gas
\item Physical vs chemical changes
\item Conservation of Energy
\item Conservation of Mass
\item Scientific notation e.g. $164.23 = 1.6423 \times 10^2$
\item[] \textbf{Significant figures}
  \begin{itemize}
  \item What do significant figures imply?
  \item Leading, sandwiched, and trailing zeroes
  \item Rounding rules for multiplying, division, addtion and substraction
  \item Combining multiple steps
  \end{itemize}
\item Unit conversion and prefixes
\item Scientific method and examples where scientific method is applied
\end{itemize}

\textbf{Chapter 2: Atoms, Ions, and the Periodic Table}

\begin{itemize}
  \setlength\itemsep{0em}
\item Dalton's Atomic Theory
\item Law of definite proportions
\item What are atoms made of?
\item Millikan's oil-drop experiment
\item[] \textbf{J.J. Thompson}
  \begin{itemize}
  \item Cathode-ray tube experiment
  \item Plum Pudding Model
  \end{itemize}
\item Isotopes, atomic number, and mass number
\item What are ions?
\item Mass spectrometer
\item Relative atomic mass calculation
\item Periodic Table and its classifications
\end{itemize}

\textbf{Chapter 3: Chemical Compounds}

\begin{itemize}
  \setlength\itemsep{0em}
\item Classifying ionic and molecular compounds
\item Familiarize with the periodic table symbols
  and memorize polyatomic ions
\item Understand the oxidation states for elements
\item Naming rules for ionic and molecular compounds
\item Naming acids
\end{itemize}

\textbf{Chapter 4: Chemical Composition}

\begin{itemize}
  \setlength\itemsep{0em}
\item Mass percent composition formula
\item The concept of the mol (Avogadro's number)
\item Finding molar masses
\item Molarity (mol/L)
\item Dilutions ($M_1V_1 = M_2V_2$)
\end{itemize}

\textbf{Chapter 5: Stoichiometry}

\begin{itemize}
  \setlength\itemsep{0em}
\item Chemical equations
\item Mole ratios - converting from one compounds to another
\item Meaning of mole ratios
\item Limiting reagent problems
\item Theoretical yield and percent yield
\item Molarity (mols/L)
\item Dilution problems ($M_1V_1 = M_2V_2$)
\item Molarity of ions
\item Stoichiometry with molarity
\end{itemize}

\textbf{Chapter 6: Thermochemistry}

\begin{itemize}
  \setlength\itemsep{0em}
\item Kinetic energy vs potential energy
\item Sign conventions (+/-)
\item Internal energy - work and heat
\item State function
\item Endothermic, exothermic reactions and effects of catalysts
\item Calculating heat ($q=mc\Delta T$) and thermal equilibrium
\item Calorimetry calculations
\item Standard enthalpy and enthalpy of reaction
  \item Hess' Law
\end{itemize}

\textbf{Chapter 7: Gases}

\begin{itemize}
  \setlength\itemsep{0em}
\item Boyle's Law
\item Charles' Law
\item Avogadro's Law
\item Assumptions of ideal gas law
\end{itemize}

\textbf{Chapter 8: The Quantum Model of the Atom}

\begin{itemize}
  \setlength\itemsep{0em}
\item Relationship wavelength $\lambda$, frequency $\nu$, and speed of light $c$
\item Relationship between light energy and frequency and wavelength
\item Electromagtic spectrum (Radio waves, Microwaves, IR, visible light, UV vis, X-ray, and gamma rays)
\item Bohr Model of the atom and its limitation
\item Rydberg equation
\item Quantum numbers ($n, l, m_l, m_s$) and atomic orbitals
\item Heisenberg Uncertainty principle, Pauli Exclusion principle, Aufbau principle, and
  Hund's rule
\item Electron configurations (long and short handed)
\end{itemize}

\textbf{Chapter 9: Periodic trends}

\begin{itemize}
  \setlength\itemsep{0em}
\item Valence and core electrons
\item Atomic and ionic radius
\item Electronegativity trends
\item Ionization Energy and Electron Affinity
\end{itemize}

\textbf{Chapter 10: Covalent Bonding}

\begin{itemize}
  \setlength\itemsep{0em}
\item Lewis structure and steps to draw compounds
\item Octet rule and exceptions to the octet rules
\item Resonance structures and hybrid structure
\item Nonpolar/Polar bonds and electronegativity
\end{itemize}

\textbf{Chapter 11: Molecular Shape}

\begin{itemize}
  \setlength\itemsep{0em}
\item Define VSEPR Model
\item Electronic arrangement/structure and Geometric Structure
\item Nonpolar/Polar molecules
\end{itemize}


\end{document}
